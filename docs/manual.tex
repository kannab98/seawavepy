\documentclass[a4paper,14pt]{extarticle}

\usepackage[utf8]{inputenc}
\usepackage[T2A]{fontenc}
\usepackage[english]{babel}

\usepackage[mode=buildnew]{standalone}
\usepackage{setspace}


% Различные пакеты
\usepackage{
	amssymb, amsfonts, amsmath, amsthm, physics,
	cancel, indentfirst,makecell,multirow, 
	graphicx, tikz, mathtools, float, setspace,caption,subcaption
} 

\usepackage{mathtools}

% Эта опция включает нумерацию только у тех формул,
% на которые есть ссылка в документе
\mathtoolsset{showonlyrefs=true} 

 % Цвета для гиперссылок
\definecolor{linkcolor}{HTML}{000000} % цвет ссылок
\definecolor{urlcolor}{HTML}{799B03} % цвет гиперссылок
 
\usepackage{xcolor}
\usepackage[
    unicode, 
    colorlinks, 
    urlcolor=urlcolor, 
    linkcolor=linkcolor,
    citecolor=linkcolor
]{hyperref}

% Увеличенный межстрочный интервал, французские пробелы
\linespread{1.2} 
\frenchspacing 

%%%%%%%%%%%%%%%%%%%%%%%%%%%%%%
%  Пользовательские команды  %
%%%%%%%%%%%%%%%%%%%%%%%%%%%%%%


\makeatletter
    \newcommand{\fftStar}[1]{\mathfrak{F}^*\qty[#1]}
    \newcommand{\fftNoStar}[1]{\mathfrak{F}\qty[#1]}
    \newcommand{\fft}{
                 \@ifstar
                 \fftStar%
                 \fftNoStar%
    }
\makeatother

\makeatletter
    \newcommand{\ifftNoStar}[1]{\mathfrak{F}^{-1}\qty[#1]}
    \newcommand{\ifftStar}[1]{\qty(\mathfrak{F}^{-1}\qty[#1])^*}
    \newcommand{\ifft}{
                 \@ifstar
                 \ifftStar%
                 \ifftNoStar%
    }
\makeatother

\newcommand{\mean}[1]{\langle#1\rangle}
\newcommand\ct[1]{\text{\rmfamily\upshape #1}}
\newcommand*{\const}{\ct{const}}
\renewcommand{\phi}{\varphi}
\renewcommand{\epsilon}{\varepsilon}
%\renewcommand{\sigma}{\varsigma}


\captionsetup{subrefformat=parens}

\usepackage{array}
\usepackage{pstool}


% Диагональная ячейка в таблице ( типа |a/b|)
\newcolumntype{x}[1]{>{\centering\arraybackslash}p{#1}}
\newcommand\diag[4]{%
  \multicolumn{1}{p{#2}|}{\hskip-\tabcolsep
  $\vcenter{\begin{tikzpicture}[baseline=0,anchor=south west,inner sep=#1]
      \path[use as bounding box] (0,0) rectangle (#2+2\tabcolsep,\baselineskip);
      \node[minimum width={#2+2\tabcolsep},minimum height=\baselineskip+\extrarowheight] (box) {};
      \draw (box.north west) -- (box.south east);
      \draw (box.south west) -- (box.north west);
      \node[anchor=south west] at (box.south west) {\footnotesize#3};
      \node[anchor=north east] at (box.north east) {\footnotesize#4};
 \end{tikzpicture}}$\hskip-\tabcolsep}}

%%%%%%%%%%%%%%%%%%%%%%%%%%%%%
%  Геометрия и колонтитулы  %
%%%%%%%%%%%%%%%%%%%%%%%%%%%%%


\usepackage{geometry}
\geometry       
    {
        left            =   2cm,
        right           =   2cm,
        top             =   2.5cm,
        bottom          =   2.5cm,
        bindingoffset   =   0cm
    }

% Настройка содержания, точки после нумераций
\usepackage{tocloft}
\addto\captionsrussian{\renewcommand{\contentsname}{Оглавление}}
\renewcommand{\cftsecleader}{
	\cftdotfill{\cftdotsep}}
% \renewcommand{\thesection}{
	% \arabic{section}.}
% \renewcommand{\thesubsection}{
	% \arabic{section}.\arabic{subsection}.}
% \renewcommand{\thesubsubsection}{
	% \arabic{section}.\arabic{subsection}.\arabic{subsubsection}.}     
\usepackage[explicit]{titlesec}

% Колонтитулы
%\usepackage{fancyhdr} 
	%\pagestyle{plain} 
	%\fancyhead{} 
	%\fancyhead[R]{} 
	%\fancyhead[L]{} 
	%\fancyfoot{} 
	%\fancyfoot[C]{\thepage} 


%\include{preamble/python.tex}

\setcounter{secnumdepth}{0}
\title{SeawavePy -- Sea Surface Simulation}
\author{Ponur K.A.}
\date{}


\newcommand{\python}{\textbf}
\begin{document}
\maketitle

\section{Surface Module Description}
Sea surface eleveations can be calculated as the sum of the harmonics with
deterministic amplitudes and random phases
\begin{gather}
    \label{eq:surface2d}
    \xi(\vec r,t) = \sum\limits_{n=1}^{N} \sum\limits_{m=1}^{M}
    A_{n} \cdot
    F_{nm} \cos \qty(\omega_n t + \vec \kappa_{nm} \vec r + \psi_{nm}), \\
    A_n(\kappa_n) = \frac{1}{2 \pi} \sqrt{\int\limits_{\Delta \kappa_n} 
        S_\xi(\kappa)
    \dd \kappa}, \\
    F_{nm}(\kappa_n,\phi_m) = \sqrt{\int\limits_{\Delta \phi_m}
    \Phi_{\xi }(\kappa_n,\phi) \dd \phi},
\end{gather}
where $A_{nm}(\kappa)$ -- wave amplitude calculated from one-dimensional
wave spectrum $S_\xi(\kappa)$,  $\vec r$ --
radius measured from zero sea level,  $\psi$ -- random phases,  $\Phi$ -- azimuthal wave distribution.

Knowing the elevations, we can calculate the slopes  of the surface


\begin{equation}
    \label{eq:slopes}
    \begin{aligned}
        \sigma_x(\vec r,t) & = \pdv{\xi}{x} \\
        \sigma_y(\vec r,t) & = \pdv{\xi}{y} \\
        \sigma_z(\vec r,t) & = \pdv{\xi}{z} =  1 
    \end{aligned}
\end{equation}
The function \python{wind} in module \python{core.surface} calculates equations \eqref{eq:surface2d},
\eqref{eq:slopes} on the GPU and returns an array of the NetCDF format
with fields:
\begin{itemize}
    \item \python{elevations} -- elevations of surface with dimensions
        $(x,y,t)$
    \item \python{slopes} -- slopes of surface with dimensions
        $(3, x,y,t)$
    \item \python{velocities}
    -- orbital velocities of surface with dimensions
        $(3, x,y,t)$
    \item \python{spectrum} -- Two dimensional wave spectrum
\end{itemize}

\section{Tilt-modulation}%
\label{sec:tilt_modulation}


The equation \eqref{eq:slopes} is the normal vector at a point on the surface
\begin{equation}
    \label{eq:}
    \vec n = \frac{\vec i \cdot \sigma_x+ \vec j \sigma_y + \vec k \cdot 1 }{\sqrt{\sigma_x^2 +
    \sigma_y^2 + 1}}
\end{equation}
\begin{equation}
    \label{eq:}
    \vec u = (x,y,\xi)
\end{equation}

Then we can calculate tilt-modulation effect 
\begin{equation}
    \label{eq:}
    \sigma_{tilt}(x,y,t) = \begin{cases}
        \vec n \vec u, &\text{ if } \vec n \vec u > 0\ \\ 
        0,  & \text{ if } \vec n \vec u \leq 0
        
    \end{cases} 
\end{equation}

These equations are a complete replacement for equations (3-6) from the article
"Research of X-band Radar Sea Clutter Image Simulation Model"


\end{document}


